\documentclass[11pt]{article}
\usepackage[latin1]{inputenc}
\usepackage{amsmath}
\usepackage{epsfig}
\usepackage{pgf}
\usepackage{graphicx}
 \usepackage{amssymb}
\usepackage{url}
\def\urltilda{\kern -.15em\lower .7ex\hbox{\~{}}\kern .04em}

    \oddsidemargin  0.0in
    \evensidemargin 0.0in
    \textwidth      6.5in
    \headheight     0.0in
        \headsep        0.0in
    \topmargin      0.0in
    \textheight=9.0in

\begin{document}
\title{Lecture 7, Exercise 1}
\maketitle
\author{Author: Alan Le Goallec}

\section{Question} The Worst Case of Perceptron\\
Source: Shalev-Shwartz, Shai, and Shai Ben-David. Understanding machine learning: From theory to algorithms.\\
For any positive integer m, find a sequence of examples $\{(x_1, y_1), ..., (x_m, y_m)\}$ such that when running the perceptron on this sequence of examples starting from w(0) = 0, it makes at least m updates before converging.\\

Hint: Try to use training data drawn from $R^m, i.e., x_i \in R^m$.


\section{Solution}
Let us construct the following data: for every i, we let $x_i = e_i$, the standard basis of i in $R^m$.\\
First note that if we pick $w = (y_1,y_2,...,y_m)^T$, we have for any i, $y_ix_i^Tw = y_ie_i^Tw =y_i^2 =1 >0$.\\
So this example is indeed linearly separable and the perceptron converges to a $w^*$.\\
When the perceptron converges we have $y_i x_i^Tw^* >0$,for every i. \\
Since $y_i$ is either 1 or -1, we have that for any i, $x_i^Tw^* =e_i^T w^* =(w^*)_i \neq 0$.\\
That is, the solution $w^*$ after running the perceptron algorithm must be nonzero for every coordinate.\\
Note that when running perceptron, we update w when there exists i such that $y_ix_i^T w(t) < 0$, and w is updated as follows:\\
$w(t+1) = w(t) + \eta y_ix_i$.\\
Since $x_i = e_i$, it is immediate that the above update only changes the ith coordinate of w(t). Note that initially, w(0) = 0. Hence, in order to update it to become a vector such that each of its coordinate is nonzero, we need at least m updates. Therefore, the perceptron needs at least m updates to converge.

\end{document}